\documentclass[a4paper,12pt]{article}
\usepackage[utf8]{inputenc}
\usepackage[ngerman]{babel}
\usepackage{lmodern}
\usepackage{enumitem}
\usepackage{titlesec}
\usepackage{hyperref}
\usepackage{geometry}
\geometry{margin=2.5cm}

\titleformat{\section}{\normalfont\Large\bfseries}{\thesection}{1em}{}
\titleformat{\subsection}{\normalfont\large\bfseries}{\thesubsection}{1em}{}

\title{\textbf{Dokumentation – JobTinder Webanwendung}}
\author{Webtechnologien 2 – Praktikumsauftrag}
\date{\today}

\begin{document}

\maketitle

\section{Produktvision}

Unsere Webanwendung \textbf{JobTinder} verfolgt das Ziel, den Bewerbungsprozess für junge Jobsuchende so einfach, intuitiv und mobilfreundlich wie möglich zu gestalten. Ähnlich wie bei bekannten Dating-Apps können Jobsuchende durch Stellenausschreibungen „swipen“, um schnell geeignete Angebote zu entdecken. Die Benutzeroberfläche ist asynchron gestaltet, sodass Jobsuchende und Jobgebende jeweils eine eigene, angepasste Nutzererfahrung erhalten.

Zusätzlich bieten wir vorgefertigte Lebenslauf-Layouts sowie einfache Stellenvorlagen an, um den Aufwand für beide Seiten so gering wie möglich zu halten. Unser Fokus liegt auf Schnelligkeit, Übersichtlichkeit und einem klaren, modernen Interface.

\section{Personas}

\subsection*{Persona 1: Leon (22 Jahre)}
\textbf{Berufseinsteiger}, sucht nach einem Praktikum oder Nebenjob. Er will sich nicht durch komplizierte Formulare kämpfen und bevorzugt mobile Anwendungen mit schneller Bedienung.

\subsection*{Persona 2: Julia (30 Jahre)}
\textbf{HR-Managerin} in einem Start-up. Sie möchte ohne Umwege passende Kandidat*innen entdecken und Stellenangebote unkompliziert online stellen.

\subsection*{Persona 3: Amir (27 Jahre)}
\textbf{Freelancer auf Jobsuche}. Ihm ist Flexibilität wichtig, er sucht projektbasierte Jobs und möchte sich mit minimalem Aufwand bewerben können.

\section{User Stories}

\begin{itemize}[leftmargin=1.5cm]
    \item \textbf{Als Leon} möchte ich durch offene Stellenangebote swipen können, um schnell relevante Jobs zu entdecken.
    \item \textbf{Als Julia} möchte ich mithilfe einer Vorlage schnell neue Jobangebote online stellen, ohne technische Vorkenntnisse.
    \item \textbf{Als Amir} möchte ich einen Lebenslauf automatisch generieren lassen, damit ich keine Zeit mit Formatierung verliere.
\end{itemize}

\section{Epics}

\begin{itemize}[leftmargin=1.5cm]
    \item \textbf{Nutzerverwaltung}: Registrierung, Login, Rollen (Jobsuchende vs. Jobgebende)
    \item \textbf{Matching-System}: Swipe-Mechanik zur Navigation durch Stellenanzeigen bzw. Kandidaten
    \item \textbf{Asynchrone UI}: Unterschiedliche Inhalte für Jobsuchende und Jobgebende
    \item \textbf{Vorlagen und Layouts}: Fertige Designs für Lebensläufe und Stellenausschreibungen
    \item \textbf{Benachrichtigungen}: Push- oder E-Mail-Benachrichtigungen bei Matches oder Nachrichten
    \item \textbf{Verwaltung von Stellenausschreibungen}: Erstellen, Bearbeiten, Deaktivieren
\end{itemize}

\section{Repository Integration (Git)}

Die gesamte Projektstruktur sowie diese Dokumentation werden in ein Git-Repository eingecheckt. Dabei wird eine klare Struktur eingehalten:

\begin{itemize}
    \item \texttt{README.md} für erste Projektübersicht
    \item \texttt{/docs} Ordner für LaTeX-Dokumentation
    \item \texttt{/frontend} und \texttt{/backend} getrennt
\end{itemize}

\noindent Beispielhafte Git-Kommandos:

\begin{verbatim}
git init
git add .
git commit -m "Add initial documentation and project structure"
git push origin main
\end{verbatim}

\end{document}

